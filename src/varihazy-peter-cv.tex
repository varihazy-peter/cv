%----------------------------------------------------------------------------------------
%    PACKAGES AND OTHER DOCUMENT CONFIGURATIONS
%----------------------------------------------------------------------------------------

\documentclass[9pt]{developercv} % Default font size, values from 8-12pt are recommended

%----------------------------------------------------------------------------------------
\usepackage[none]{hyphenat}

\begin{document}

%----------------------------------------------------------------------------------------
%    TITLE AND CONTACT INFORMATION
%----------------------------------------------------------------------------------------

\newcommand{\iconl}[3]{% The first parameter is the FontAwesome icon name, the second is the box size and the third is the text
    \vcenteredhbox{\textcolor{black}{#3}}% Text
    \hspace{0.2cm}% Whitespace
    \vcenteredhbox{\colorbox{black}{\makebox(#2, #2){\textcolor{white}{\large\csname fa#1\endcsname}}}}% Icon and box
}
\begin{minipage}[t]{0.5\textwidth} % 45% of the page width for name
    \vspace{-\baselineskip} % Required for vertically aligning minipages
    
    % If your name is very short, use just one of the lines below
    % If your name is very long, reduce the font size or make the minipage wider and reduce the others proportionately
    \colorbox{black}{{\HUGE\textcolor{white}{\textbf{Péter}}}} % Given name
    
    \colorbox{black}{\HUGE\textcolor{white}{\textbf{Variházy}}} % Family name
    
    \vspace{6pt}
    
    {\huge Senior Java Developer} % Career or current job title
\end{minipage}
\begin{minipage}[t]{0.5\textwidth} % 27.5% of the page width for the first row of icons
    \begin{flushright}
    \vspace{-\baselineskip} % Required for vertically aligning minipages
    \iconl{MapMarker}{12}{Pécs, Hungary} \\
    \iconl{Phone}{12}{\href{tel:+36706374778}{+36 70 637 4778}} \\
    \iconl{Facebook}{12}{\href{https://www.facebook.com/peter.varihazy}{peter.varihazy}} \\
    \iconl{Linkedin}{12}{\href{https://www.linkedin.com/in/peter-varihazy/}{peter-varihazy}} \\
    \iconl{At}{12}{\href{mailto:varihazy.peter@gmail.com}{varihazy.peter@gmail.com}} \\
    \end{flushright}
\end{minipage}

%----------------------------------------------------------------------------------------
%    INTRODUCTION, SKILLS AND TECHNOLOGIES
%----------------------------------------------------------------------------------------

\cvsect{Who Am I?}

\begin{minipage}[t]{0.48\textwidth} % 40% of the page width for the introduction text
    \vspace{-\baselineskip} % Required for vertically aligning minipages
    From an early age I liked creating things. I used to play with lego, draw maps and build sand castles.
    So choosing programming as my career was a pretty obvious choice.
    During my work I touched many fields, ranging from industrial farming to VAT reporting. \\

    \textbf{My motto}:
    You cannot understand everything.
    You cannot achieve perfection in anything.
    Attempt it, and you can see the world and yourself in different ways. \\

    In the last couple of years I started to live and work by some patterns or philisophies that are both good for me and for my firm. \\
\end{minipage}
\hfill % Whitespace between
\begin{minipage}[t]{0.5\textwidth} % 50% of the page for the skills bar chart
    \vspace{-\baselineskip} % Required for vertically aligning minipages

    \textbf{Simplicity} -- We developers are working (mostly) in teams, so one of the most important factor of any code it's readibility. \\
    % \textbf{Twelve-Factor App} -- Simple constraints, that can make development, deployment and testing easier. \\
    \textbf{Do not reinvent the wheel} -- If I need an ORM I use one,
    if I need a csv parser I pick one,
    if I need a threadsafe or immutable collection I import an implementation.
    My job is to solve problems, create bussiness value, not to produce huge amount of Java code.\\
    \textbf{Test Driven Development (TDD)} -- I like to sleep well and not worry about the leatest deployment or any fresh changes.
    Therefore I write tests, and use them to speed up my development process. \\
    \textbf{Continuous integration} -- Keeping the development cycle short and working on small steps with well divided tasks can make my work more reliable. No more painful merge with conflicts.\\
\end{minipage}

%----------------------------------------------------------------------------------------
%    EXPERIENCE
%----------------------------------------------------------------------------------------

\cvsect{Experience}

\begin{entrylist}
    \entry
        {2022 \\ Remote}
        {\href{https://www.ambergrid.lt/en/}{Amber Grid} | \href{https://en.wikipedia.org/wiki/Ukrenergo}{Ukrenergo} | \href{https://en.wikipedia.org/wiki/Electricity_Authority_of_Cyprus}{Electricity Authority of Cyprus}}
        {\href{https://navitasoft.com/en}{Navitasoft}}
        {ETRM -- Energy Trading and Risk Management (tool). \\
            \texttt{Kafka}\slashsep
            \texttt{''Microservices''}\slashsep
            \texttt{JUnit (4)}\slashsep
            \texttt{OpenAPI}\slashsep
            \texttt{Angular}\slashsep
            \texttt{TypeScript}
        }
    \entry
        {2021 \\ Pécs, Hungary}
        {BMW -- Zedis}
        {\href{https://cubicfox.com/}{Cubicfox}}
        {A diagnostic tool for different parts for BMW components.\\
            My main responsibility was to develop UI based tests for the Angular frontend.\\
            \texttt{Java Selenium}\slashsep
            \texttt{Jenkins}
        }
    \entry
        {2021 \\ Pécs, Hungary}
        {BMW -- OTD (Official Technical Data)}
        {\href{https://cubicfox.com/}{Cubicfox}}
        {The software provides tailored technical specification based on vehicle configurations. \\
            \texttt{Java EE}\slashsep
            \texttt{Oracle SQL}
        }
    \entry
        {2017 -- 2021 \\ Pécs, Hungary}
        {\href{https://app.hellotax.com}{Hellotax -- VatJar}}
        {Cubicfox}
        {A web application that supports European Amazon sellers in their work by automating/supporting their VAT returns.
            The entire VAT calculation mechanism have to deal with different government obligations and millions of transactions. \\
            \texttt{Java}\slashsep
            \texttt{Spring}\slashsep
            \texttt{SQL}\slashsep
            \texttt{TDD}\slashsep
            \texttt{GCP}\slashsep
            \texttt{Amazon MWS}\slashsep
            \texttt{PHP}
        }
    \entry
        {2017 \\ Attala, Hungary}
        {Logistics portal}
        {Freelancer}
        {A system that is responsible for monitoring and organizing the lifecycle of products, which includes delivery,
            transmission, modification and reception of goods.\\
            Not finished / Demo \\
            \texttt{PHP}\slashsep
            \texttt{Laravel}\slashsep
            \texttt{MariaDB}
        }
    \entry
        {2015 -- 2017 \\ Attala, Hungary}
        {Automatic controller unit}
        {Freelancer}
        {Constructed the functional operator of a pancake vending machine and integrated the graphical interface. Created the
            offline and online registry for stocks and finances as well as the event log.\\
            \texttt{Python}\slashsep
            \texttt{IoT}
        }
    \entry
        {2014 -- 2016 \\ Attala, Hungary}
        {Growatt solar panel logger server and proxy}
        {Freelancer}
        {Composed a secondary server for the Growatt Shine hardwares (WebBox, WiFiBox, Lan) as their database and web
            interface were insufficient for their needs. There was no option for a secondary server integration at first, therefore the
            secondary server had to be created in such a way that it would meet the requirements of a modbus-like tcp protocol.
            The server was responsible for the communication between the Growatt server and the hardwares, whilst simulating a
            direct connection. \\
            \texttt{Python}\slashsep
            \texttt{SQL}\slashsep
            \texttt{TCP socket}
        }
    \entry
        {2014 -- 2016 \\ Attala, Hungary}
        {Control panel that communicates with the alarm center (“PLC”)}
        {Freelancer}
        {Worked on the PLC controller of a large building, which granted the option for online monitoring, greatly detailed
            journaling and control. \\ 
            \texttt{Python}\slashsep
            \texttt{IoT}\slashsep
            \texttt{I2C}\slashsep
            \texttt{1-Wire}\slashsep
            \texttt{Yocto}
        }
    \entry
        {2015 \\ Attala, Hungary}
        {Growatt remote display for inverters}
        {Freelancer}
        {Created a HDMI compatible device for the advertising of the Growatt company’s products. This device is responsible
            for the visual representation of active data.
            \\ \texttt{Python}\slashsep\texttt{IoT}\slashsep\texttt{Raspberry Pi}}
    \entry
        {2014 \\ Attala, Hungary}
        {BeepButton access control subsystem}
        {Freelancer}
        {An access control system randomly granting access (with adjustable probability), that can be overridden remotely,
            equipped also with an additional dead man’s switch function.\\
            \texttt{Python}\slashsep
            \texttt{IoT}\slashsep
            \texttt{Raspberry Pi}
        }
    \entry
        {2014 \\ Attala, Hungary}
        {Alarm events online display and remote control}
        {Freelancer}
        {Journaling and mediation of serial port Texecom alarms. Implementation of a web-based virtual keyboard.\\
            \texttt{Python}\slashsep
            \texttt{IoT}\slashsep
            \texttt{rs232}
        }
    \entry
        {2013 \\ Pécs, Hungary}
        {Internship}
        {Intern --- HC Linear}
        {Data uploading, web-based route planner development, Manual composition. During this internship, I had to rebuild a
            part of an existing, but outdated map-based route planner, and a fleet management web server. The refactored
            softwares had to be connected to an improved database.\\
            \texttt{JS}
        }
\end{entrylist}

%----------------------------------------------------------------------------------------
%    ADDITIONAL INFORMATION
%----------------------------------------------------------------------------------------

\begin{minipage}[t]{0.49\textwidth}
    \vspace{-\baselineskip} % Required for vertically aligning minipages

    \cvsect{Tools}

    \textbf{Linux}
    -- I have been using Arch linux as my only OS for more than 10 years.

    \textbf{Docker} 
    -- Because I configured multiple VMs durings my career, I prefer using docker.

    \textbf{Git}
    -- Whether I am working alone or in a team I always appreciate git, it saved me a couple of times.

    \textbf{Jira, confluence, BitBucket}
    -- I have been using Attlassian products since 2019.

    \textbf{Maven}
    -- From running tests to formating the source code.

    \textbf{Eclipse}
    -- I used Pycharm, NetBeans, VS code, but my default choice (if it is possible) is Eclipse.

    \textbf{Google Cloud Platform (GCP)}
    -- I have been using GCP since 2019. I have experience with \textbf{Kubernetes}. However, I never had the chance to use it in production environment.

    \cvsect{Languages}

    \textbf{English}
     - My English is far from good. I have daily stand-up meetings and multiple hour conference calls to collaborate with an international company. I have no certificate.\\
    \textbf{Hungarian}
     - Native language.

\end{minipage}
\hfill
\begin{minipage}[t]{0.49\textwidth}
    \vspace{-\baselineskip} % Required for vertically aligning minipages

    \cvsect{programming language, Framework, Library}

    \textbf{Java} -- I have been developing in Java since 2019.
        I like the lambda functions and \texttt{@FunctionalInterface} because it makes the composition easier, less verbose.
        With the immutable objects and collections I do not need to worry about data change and makes the debug easier.
        The \texttt{CompletableFuture} makes the async code simpler, easier to read.
        I love the \texttt{Stream} which can use the mentined ones and help separat your code in smaller peaces.

    \textbf{Spring Framework} -- When it comes to Java, choosing Spring was easy.
    This framework reduces the amount of code. 
    The IoC makes my code more reliable and test friendly. Allows me to concentrate on business logic.
    

    \textbf{JUnit} -- Because it is very well integrated with spring I write my tests in JUnit 5.

    \textbf{Testcontainers} -- It reduces the gap between dev(/test) and production enviroment.

    \textbf{Lombok} -- It removes the trivial getters, setters, toString and others. With the \texttt{@Value} annotation a DTO class is as short as it can be. I always add it to the pom.

    \textbf{SQL} -- I have been using MySQL and PostgreSQL for more than 10 years.
    Because of the version control I tend to place my queries in source code instead of the stored procedures or views. I have been gathering some Oracle db experience for a little over a year.

    \textbf{JPA 2.1 and Hibernate} -- Simple persistence layer, happy programmer.

\end{minipage}

\cvsect{Interests, Hobbies}

\textbf{Learning} -- I am curious. I like to learn. History, biology, other sciences, cooking, of course programming, and others.

\textbf{Serverless} -- With serverless solutions we sacrifice some control, but the benefits we gain and the new constraints are interesting topics for me. K8S, cloud run and not traditional databases are fascinates me.

\textbf{Series, Books, Comedy} -- I like to watch series, films and read books — comedies and dramas.

\textbf{Cooking} -- I used to cook not just for myself but also for my friends.
%----------------------------------------------------------------------------------------

\end{document}